flexfringe (formerly D\+F\+A\+S\+AT), a flexible state-\/merging framework written in C++.

\subsubsection*{What is this repository for?}

You can issue pull requests for bug fixes and improvements. Most work will happen in the development branch while master contains a more stable version.

\subsubsection*{How do I get set up?}

flexfringe has one required dependency\+: libpopt for argument parsing. Some heuristic functions bring their own dependencies. We provide an implementation of a likelihood-\/based merge function for probabilistic D\+F\+As. It needs the G\+NU scientific library (development) package (e.\+g. the libgsl-\/dev package in Ubuntu).

If you want to use the reduction to S\+AT and automatically invoke the S\+AT solver, you need to provide the path to the solver binary. flexfringe has been tested with lingeling (which you can get from \href{http://fmv.jku.at/lingeling/}{\tt http\+://fmv.\+jku.\+at/lingeling/} and run its build.\+sh).

You can build and compile the flexfringe project by running

\$ make clean all

in the main directory to build the executable named /flexfringe/.

\subsubsection*{How do I run it?}

Run ./flexfringe --help to get help.

The start.\+sh script together with some .ini files provides a shortcut to storing

Example\+:

\$ ./start.sh mealy-\/batch.\+ini data/simple.\+traces

See the .ini files for documentation of parameter flags.

\subsubsection*{Output files}

flexfringe will generate several .dot files into the specified output directory (. by default)\+:


\begin{DoxyItemize}
\item pre\textbackslash{}\+:$\ast$.dot are intermediary dot files created during the merges/search process.
\item final.\+dot is the end result
\end{DoxyItemize}

You can plot the dot files via

\begin{quote}
\$ dot -\/\+Tpdf file.\+dot -\/o outfile.\+pdf \end{quote}
or \begin{quote}
\$ ./show.sh final.\+dot \end{quote}


after installing dot from graphviz.

\subsubsection*{Contribution guidelines}


\begin{DoxyItemize}
\item Fork and implement, request pulls.
\item You can find sample evaluation files in ./evaluation. Make sure to R\+E\+G\+I\+S\+T\+ER your own file to be able to access it via the -\/h flag.
\end{DoxyItemize}

\subsubsection*{Who do I talk to?}


\begin{DoxyItemize}
\item Sicco Verwer (original author; best to reach out to for questions on batch mode and S\+AT reduction)
\item Christian Hammerschmidt (author of the online/streaming mode and interactive mode) 
\end{DoxyItemize}